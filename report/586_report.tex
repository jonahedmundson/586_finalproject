\documentclass[11pt, a4paper]{article}
\usepackage[utf8]{inputenc}
\usepackage[margin=1in]{geometry} %Sets proper 1-inch margins. 
\usepackage{amsmath} %Only load this if you are using math/equations.
\usepackage{graphicx} %Only need to call this if inserting images.
\usepackage{caption} %Only need to call this if inserting captions.
\usepackage{float} %Allows the use of the [H] specifier. 
%\graphicspath{{C:/Users/jonah/Pictures/}} %Sets the working directory for images.
\usepackage[colorlinks,citecolor=blue,linkcolor=blue,urlcolor=blue]{hyperref} %Allows for the embedding of urls. 
\usepackage{setspace}
\usepackage{blindtext}

\pagenumbering{arabic}

%\usepackage{fontspec} %%in order for this font stuff to work, you must compile using xelatex+makeindex+bibtex (or at minimum xelatex)
%\setmainfont[Mapping=tex-text-ms]{Essays1743}

\usepackage{fancyhdr}

\pagestyle{fancy}
\fancyhf{}
\rhead{Greenough \& Edmundson \\ 2023}
\lhead{\thepage}

\newcommand{\comment}[1]{}

\usepackage{Sweave}
\begin{document}
\Sconcordance{concordance:586_report.tex:586_report.Rnw:%
1 26 1 1 0 52 1}


\begin{center}
\LARGE{\textsc{Machine Learning Prediction of Political Orientation using Facebook Profile Pictures}}
\par
\vspace{1pc}
\par
\Large{DATA586 Project Report}
\par
\vspace{1.0pc}
\par
\normalsize{Madison Greenough \& Jonah Edmundson}
\end{center}


\vspace{0.917 pc} %Creates a paragraph line break. 

%\pagebreak

%\tableofcontents

%\pagebreak
\section{Introduction}

The dataset constructed for this analysis was obtained from Facebook. Facebook profile IDs were scraped from new members of 2 Facebook ``Groups'': republicans from \href{https://www.facebook.com/groups/RU4TX/}{Republicans in Texas}, and democrats from \href{https://www.facebook.com/groups/111188856119651/}{Democratic Voices for Biden/Harris 2024}. The python API wrapper \href{https://github.com/kevinzg/facebook-scraper}{\texttt{facebook-scraper}} was used to obtain URLs to each user's profile picture. Then, profile pictures were scanned for faces and cropped to a common size of 120 x 120 pixels using the Haar Cascades facial detection algorithm. 


Finally, faces were assigned X facial attributes using Y.





Assumptions of the following analysis due to the method of dataset construction are as follows:

\begin{itemize}
\item ``Democrats'' and ``republicans'' are good proxies for ``liberal'' and ``conservative'', respectively. 
\item Democrats/republicans on Facebook are representative of democrats and republicans in general. 
\item Users that set profiles pictures of themselves versus of a non-face picture do not have different facial attributes. 
\end{itemize}











\end{document}
